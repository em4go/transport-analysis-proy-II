% Options for packages loaded elsewhere
\PassOptionsToPackage{unicode}{hyperref}
\PassOptionsToPackage{hyphens}{url}
%
\documentclass[
  10pt,
]{article}
\usepackage{amsmath,amssymb}
\usepackage{iftex}
\ifPDFTeX
  \usepackage[T1]{fontenc}
  \usepackage[utf8]{inputenc}
  \usepackage{textcomp} % provide euro and other symbols
\else % if luatex or xetex
  \usepackage{unicode-math} % this also loads fontspec
  \defaultfontfeatures{Scale=MatchLowercase}
  \defaultfontfeatures[\rmfamily]{Ligatures=TeX,Scale=1}
\fi
\usepackage{lmodern}
\ifPDFTeX\else
  % xetex/luatex font selection
  \setmainfont[]{LiberationSerif}
  \setsansfont[]{LiberationSans}
  \setmonofont[]{LiberationMono}
\fi
% Use upquote if available, for straight quotes in verbatim environments
\IfFileExists{upquote.sty}{\usepackage{upquote}}{}
\IfFileExists{microtype.sty}{% use microtype if available
  \usepackage[]{microtype}
  \UseMicrotypeSet[protrusion]{basicmath} % disable protrusion for tt fonts
}{}
\makeatletter
\@ifundefined{KOMAClassName}{% if non-KOMA class
  \IfFileExists{parskip.sty}{%
    \usepackage{parskip}
  }{% else
    \setlength{\parindent}{0pt}
    \setlength{\parskip}{6pt plus 2pt minus 1pt}}
}{% if KOMA class
  \KOMAoptions{parskip=half}}
\makeatother
\usepackage{xcolor}
\usepackage[left=1.5cm,right=1.5cm,top=2cm,bottom=2cm]{geometry}
\usepackage{graphicx}
\makeatletter
\def\maxwidth{\ifdim\Gin@nat@width>\linewidth\linewidth\else\Gin@nat@width\fi}
\def\maxheight{\ifdim\Gin@nat@height>\textheight\textheight\else\Gin@nat@height\fi}
\makeatother
% Scale images if necessary, so that they will not overflow the page
% margins by default, and it is still possible to overwrite the defaults
% using explicit options in \includegraphics[width, height, ...]{}
\setkeys{Gin}{width=\maxwidth,height=\maxheight,keepaspectratio}
% Set default figure placement to htbp
\makeatletter
\def\fps@figure{htbp}
\makeatother
\setlength{\emergencystretch}{3em} % prevent overfull lines
\providecommand{\tightlist}{%
  \setlength{\itemsep}{0pt}\setlength{\parskip}{0pt}}
\setcounter{secnumdepth}{5}
\usepackage{float}
\usepackage[colorlinks]{hyperref}
\hypersetup{colorlinks,linkcolor={black},citecolor={black},urlcolor={red}}
\ifLuaTeX
  \usepackage{selnolig}  % disable illegal ligatures
\fi
\IfFileExists{bookmark.sty}{\usepackage{bookmark}}{\usepackage{hyperref}}
\IfFileExists{xurl.sty}{\usepackage{xurl}}{} % add URL line breaks if available
\urlstyle{same}
\hypersetup{
  pdftitle={Memoria},
  pdfauthor={Laiqian Ji},
  hidelinks,
  pdfcreator={LaTeX via pandoc}}

\title{Memoria}
\author{Laiqian Ji}
\date{2024-05-03}

\begin{document}
\maketitle

{
\setcounter{tocdepth}{3}
\tableofcontents
}
\newpage
\section*{Agradecimientos}

\newpage
\section*{Resumen}

\hypertarget{planteamiento}{%
\section{Planteamiento}\label{planteamiento}}

En el contexto de las ciudades modernas y específicamente de Valencia,
la comprensión de la dinámica urbana es esencial para el desarrollo
sostenible y la calidad de vida de sus habitantes. Por un lado, los
barrios, como unidades elementales de la estructura urbana, intervienen
en la experiencia diaria de sus residentes. Al mismo tiempo, el
transporte público, particularmente los sistemas de metro y bus, actúan
como infraestructura vital que conecta y facilita la movilidad dentro de
las áreas metropolitanas más allá del transporte privado.

La interacción entre ambos, ofrece un rico campo de estudio que abarca
diversos aspectos sociales, económicos y ambientales. En este contexto,
el presente proyecto se centra en la caracterización de los barrios y el
transporte público, y su relación con el sistema de metro, utilizando
para ello diferentes técnicas de análisis y algoritmos de aprendizaje
automático.

La caracterización de barrios implica la identificación y el análisis de
una amplia gama de variables, que van desde aspectos demográficos y
socioeconómicos, como la población total y la renta, hasta
características físicas y culturales, como el número de establecimientos
de hostelería, monumentos, recintos deportivos, zonas verdes y la
cantidad de estaciones de metro. Al mismo tiempo, el sistema de metro
introduce variables adicionales, como volumen promedio por hora, número
promedio de metros, número de estaciones de entrada/salida y otros
parámetros propios del estudio de grafos como la cercanía
(\textit{closeness}) y la intermediación (\textit{betweenness}). La
combinación de estos datos proporciona una visión completa de la
compleja interacción entre los barrios y el sistema de transporte
subterráneo, permitiendo identificar patrones que pueden ser de interés
para los planificadores urbanos y los responsables de la toma de
decisiones, así como a los investigadores y la sociedad en general.

Con ello, se espera que este proyecto contribuya a la comprensión de la
dinámica urbana, prestando especial atención a dos de sus componentes
más características: los barrios y el transporte público. Además, se
espera que los resultados obtenidos puedan ser de utilidad para la toma
de decisiones y la planificación urbana, así como representar una
muestra de conocimiento en el ámbito de la geografía urbana y el
análisis de redes.

\hypertarget{estado-del-arte}{%
\subsection{Estado del arte}\label{estado-del-arte}}

\hypertarget{motivaciuxf3n}{%
\subsection{Motivación}\label{motivaciuxf3n}}

\hypertarget{objetivos-de-anuxe1lisis}{%
\subsection{Objetivos de análisis}\label{objetivos-de-anuxe1lisis}}

El objetivo principal de este proyecto es alcanzar una comprensión más
profunda de los barrios urbanos y su conexión con el sistema de
transporte público. Para ello, se han desglosado cinco objetivos
específicos que se desarrollarán a lo largo del proyecto y que se
detallan a continuación:

\begin{enumerate}
\def\labelenumi{\arabic{enumi}.}
\tightlist
\item
  Caracterizar barrios
\item
  Caracterizar transporte público
\item
  Analizar la relación entre barrios y transporte público
\item
  Agrupación de barrios / Agrupación de estaciones (clustering)
\item
  Visualización
\end{enumerate}

\hypertarget{metodologuxeda}{%
\section{Metodología}\label{metodologuxeda}}

\hypertarget{etl}{%
\subsection{ETL}\label{etl}}

\hypertarget{anuxe1lisis-exploratorio}{%
\subsection{Análisis exploratorio}\label{anuxe1lisis-exploratorio}}

\hypertarget{modelos}{%
\subsection{Modelos}\label{modelos}}

\hypertarget{visualizaciuxf3n}{%
\subsection{Visualización}\label{visualizaciuxf3n}}

\hypertarget{resultados}{%
\section{Resultados}\label{resultados}}

\hypertarget{conclusiones}{%
\section{Conclusiones}\label{conclusiones}}

\hypertarget{bibliografuxeda}{%
\section{Bibliografía}\label{bibliografuxeda}}

\end{document}
